\documentclass[a4]{book}



\begin{document}

\begin{flushleft}
\hspace*{-0.5cm} \textbf{112}\hspace*{1cm} \texttt{CHAPTER EIGHT}
\end{flushleft}

\vspace*{0.5cm}
follow-up email should be sent individually, rather than sent as a "CC" to all partici-pants. The reason for this is twofold. First, the names of invitees should remain confidential, and second, the invitation to participate is much more effective when sent to each participant individually.\\
\hspace*{0.7cm} Finally, to increase the acceptance rate, Silverman (2000) suggests that creative energy be put into the wording of the topic to be studied. For example, Silverman asks us to imagine that we are being invited to a focus group, Which would you like to attend more: "Research Methods" or "New Advances in Research Methods," or "How to Conduct Research That is Cheaper, Better, and Faster," or "Ways People Have Found for Getting Beneath Surface Responses"? Further, Silverman found that when a topic is truthful and of great interest to the selected participants, participation rates routinely doubled and even tripled.\\
\hspace*{0.7cm} Once the e-researcher has obtained a sufficient number of participants and has selected the communication software, a friendly message to the participants should be sent outlining.\\

\vspace*{0.3cm}
\begin{itemize}
  \item the purpose of the study\\
  \item how confidentiality will be maintained\\
  \item how the data will be secured\\
  \item when and how the results of the study will be disseminated \\
  \item any other administrative and ethical information related to the study \\
\end{itemize}

\vspace*{0.3cm}
This process should be followed by a request for the participants to indicate that they have received, read, and understood the purpose of the study and agree to participate.\\ This process serves two function. First, it will let the e-researcher know if all the par-ticipants are connected and can successfully post and receive messages, and second, it will provide the e-researcher with informed consent from each participant. When there is no response from a group participant the e-researcher should follow up imme-diately by an email or phone call to determine the problem. The following is a sample confirmation and consent message. Note that the sample group in this example is made up of university-based teachers, thus the complex language used in the sample is appropriate for this audience but would need to be made simpler for a more gener-alized audience.\\

\vspace*{0.2cm}

\begin{center}
  \textbf{Sample Letter of Confirmation and Consent}\\
\end{center}

\vspace*{0.3cm}
\hspace*{0.7cm} Dear focus group members: \\
\vspace*{0.5cm}

\hspace*{0.7cm} Thank you for responding to my invitation to participate in this study:\\
\hspace*{0.7cm} This email provides information about the nature of the research project\\
\hspace*{0.7cm} and the procedures.\\

\vspace*{0.4cm}
\hspace*{0.7cm} The study focuses on online language practices and instructors' \\
\hspace*{0.7cm} perceptions of those practices in postsecondary institutions. The study \\
\hspace*{0.7cm} will draw on data that I have accumulated in four years of my own \\
\hspace*{0.7cm} online classroom studies. For this study, I am\\

\newpage
\begin{flushright}
 \texttt{FOCUS GROUPS} \hspace*{1cm} \textbf{113}
\end{flushright}

\vspace*{0.5cm}
endeavoring to complement work in progress (which sets out to examine discourse in the online classroom),by situating online group talk and reading within cross-curricular experiences. the intent of this part of The research is threefold:(i)to develop schemes for the analysis of student' online writing in undergraduate postsecondary inquiry and in technological problem solving;(ii)to analyze the nature of students' online writing and technological problem solving in undergranduate classes;(iii) to characterize the online reading and writing practices in underagraduate online courses.\\

\vspace*{0.4cm}
I have identified you for this study beacause you have taught an online undergraduate course. I am hoping to conduct an online focus group interview wich you to elicit your beliefs about and plans for learning in the online classroom and possibly follow up with a case study of your students' online reading and writing during one term of study.\\

\vspace*{0.3cm}
If you agree to participate. the following wil be requested of you:\\
\vspace*{0.2cm}

\begin{itemize}
  \item Respond to this email indicating that you have read understood the purpose of the study and agree to participate.\\

  \vspace*{0.2cm}

  \item participate in an online focus group for approximately 10-15 minutes per day, for 10-14 consecutive days (beginning May 7th).\\

\end{itemize}

\vspace*{0.4cm}
The focus group responses wil be confidential and all names and identifying characteristics will be removed from corresponding reports and publications. No deception will be used. The online messages connected with the study will be kept on a secured server for the duration and will be destroyed after aperiod of five years, unless permission by all participants has been granted for their further use.
 A summary of the research will be posted on the researcher's Web site at the completion of the study.I will email the URL to each participant.It is important to note that while the data will reside on a secure server and the research will make every effort to ensure data are kept safe, there is always apossibility that unauthorized access to the data my occur (e.g.,by hackers) and/or that the data my become damaged through viruses.\\

\vspace*{0.4cm}
If you agree to participate in this study, please reply to this email confirming that you have read and understood the purpose of this study.you understand your rights as a participant,and you agree to participate.Thank you. \\

\newpage
\begin{flushleft}
\hspace*{-0.5cm} \textbf{114}\hspace*{1cm} \texttt{CHAPTER EIGHT}
\end{flushleft}

\vspace*{0.5cm}
Next, the e-researcher will take steps to make the group participants feel at ease and foster communication between the participants, e-researcher, and moderator (if this person is not the e-researcher). This can be achieved by beginning with a welcome message and then posing the first question with a request for a response from all group members.\\
\hspace*{0.7cm} The following is an example of a welcome message. Again, the message should be kept short (not exceeding one screen of text) and be informal and friendly. This will set the tone for discussion, rather than for simply a sequential interview.\\

\vspace*{0.3cm}
\begin{center}
  \textbf{Sample Welcome Message}\\
\end{center}
\vspace*{0.3cm}

\hspace*{0.7cm} Dear focus group members:\\

\vspace*{0.3cm}
\hspace*{0.7cm} Welcome and thank you for agreeing to participate in this research project.\\

\vspace*{0.3cm}
\hspace*{0.7cm} Today is the first day of the online focus group, and it begins with a \\
\hspace*{0.7cm} question (seebelow). Every second day, for ten days, a new question will \\
\hspace*{0.7cm} be posted (five) questions in total). Following this will be two  \\
\hspace*{0.7cm} additional days for the "wrap-up." At some point each day you are asked\\
\hspace*{0.7cm} to respond to the question and/or respond to someone else's response.\\
\hspace*{0.7cm} As mentioned in the letter of invitation, confidentiality will be \\
\hspace*{0.7cm} maintained and only summarized information will be communicated in\\
\hspace*{0.7cm} any publication of this study.\\

\vspace*{0.2cm}
If you need to contact me during this study, I can be reached by email [email] or by phone at [area code and phone number] during the day or at day or at [area code and phone number] during the evening. If you have any technical problems, [name of technical support] will be available to help. His [her] email is [email] or he [she] can be reached at this toll free number. [area code and phone number].\\

\vspace*{0.2cm}
The following is today's question:\\

\vspace*{0.2cm}
\hspace*{0.7cm} When teaching an undergraduate online course, how do you determine \\
\hspace*{0.7cm} if a student's writing (e.g., online discussion posting) exhibits inquiry-\\
\hspace*{0.7cm} based learning, decision building, and/or problem solving?\\

\vspace*{0.2cm}
Definitions:\\

\vspace*{0.2cm}
\begin{itemize}
  \item \textbf{Inquiry-based learning} is an investigation or probe for knowledge, data, or truths.\\

  \vspace*{0.2cm}
  \item \textbf{Decision building} is arriving at a position or passing judgment on an issue that is reached after generating the alternatives, evaluating the choices, and assessing the consequences.\\

      \vspace*{0.2cm}
  \item \textbf{Problem solving} is an attempt to explain, decipher, or resolve something that is enigmatic, ambiguous, and/or cryptic.\\
\end{itemize}


\end{document}
